\documentclass[a4paper,12pt]{article}
\usepackage{amsmath}
\usepackage{amsfonts}
\usepackage{amssymb}
\usepackage{geometry}
\usepackage{bm}

\newcommand{\R}{\mathbb{R}}
\newcommand{\N}{\mathbb{N}}

% 设置页边距
\geometry{
    left=2cm,
    right=2cm,
    top=2cm,
    bottom=2cm,
}

\begin{document}

\section*{Question 1:}
Determine whether each of the following scalar-valued functions of \text{n-vectors} is linear. If it is a linear function, give its inner product representation, ie.,
an n-vector \(\bm{a}\) for which \(f(\bm{x}) = \bm{a}^T\bm{x}\) for all \(\bm{x}\). If it is not linear, give specific \(\bm{x}, \bm{y}\), \(\alpha \text{ and } \beta\) such that
\begin{align*}
    f(\alpha \bm{x} + \beta \bm{y}) \neq \alpha f(\bm{x}) + \beta f(\bm{y}).
\end{align*}

(a) The spread of values of the vector, defined as \(f(\bm{x}) = max_kx_k - min_kx_k \). \\
(b) The difference of the last element and the first, \(f(\bm{x}) = x_n - x_1\). \\

\subsection*{Answer :}
(a)  \\
Take \(\bm{x} = (1, 2, 3)\) and \(\alpha = 1, \beta = 1\) for example:
\begin{align*}
    f(\bm{x}) &= 3 - 1 = 2 \\
    f(\bm{-x}) &= -1 + 3 = 2 \\
    f(\bm{0}) &= 0 - 0 = 0 \\
    f(\bm{x} + (-\bm{x})) &= f(\bm{0}) = 0 \\
    f(\bm{x}) + f(-\bm{x}) &= 2 + 2 = 4 \\
    f(\bm{x} + (-\bm{x})) &\neq f(\bm{x}) + f(-\bm{x}) \\
\end{align*}
In conclusion, \(f(\bm{x}) = max_kx_k - min_kx_k \) is not a linear function. \\

(b) \\
We know:
\begin{align*}
    \alpha \bm{x} + \beta \bm{y} = (\alpha x_1 + \beta y_1 , \cdots , \alpha x_n + \beta y_n)
\end{align*}
\begin{align*}
    f(\bm{x}) = x_n - x_1 \\
    f(\bm{y}) = y_n - y_1 
\end{align*}
\begin{align*}
    f(\alpha \bm{x} + \beta \bm{y}) &= \alpha x_n + \beta y_n - (\alpha x_1 + \beta y_1)  \\
    \alpha f(\bm{x}) + \beta f(\bm{y}) &= \alpha(x_n - x_1) + \beta(y_n - y_1)  \\
    &= \alpha x_n + \beta y_n - (\alpha x_1 + \beta y_1) \\
    f(\alpha \bm{x} + \beta \bm{y}) &= \alpha f(\bm{x}) + \beta f(\bm{y}).
\end{align*}
Let's denote \(\bm{e}_i\) as the vector in \(\R^n\) where the i-th entry is equal to 1, and all other entries are equal to 0.

\begin{align*}
    f(\bm{x}) = \bm{a}^T\bm{x} = (\bm{e}_n - \bm{e}_1)^T\bm{x}
\end{align*}

In conclusion, \(f(\bm{x}) = x_n - x_1\) is a linear function.

\section*{Question 2:}
Consider the regression model \(y = \bm{x}^T \bm{a} + b\), where \(y\) is the predicted response,
\(\bm{x}\) is an 8-vector of features, \(\bm{a}\) is an 8-vector of coefficients, and b is the offest term.
Determine with reasoning whether each of the following statements is true or false.

(a) If \(a_3 > 0 \text{ and } x_3 > 0, \text{ then } y \geq 0\)

(b) If \(a_2 = 0\) then the prediction \(y\) does not depend on the second feature \(x_2\).

(c) If \(a_6 = -0.8\), then increasing \(x_6\) (keeping all other \(x\) is the same) will decrease \(y\).

\subsection*{Answer:}
(a) False. 

From the condition, we can deduce that \(a_3x_3 > 0\).
but we can not deduce \(\sum_{i = 1, i \neq 3}^{8}a_ix_i > 0\) and \(b > 0\).
Thus, we can not ensure \(y = \sum_{i = 1, i \neq 3}^{8}a_ix_i + b + a_3b_3 > 0\).

(b) True.

From the condition, we can deduce that \(y = \sum_{i = 1, i \neq 2}^{8} a_ix_i + b\),
which implies that \(y\) does not depend on the second feature \(x_2\)

(c) True.

Assume \(x_6' = x_6 + d, d > 0\), we know \(y' = \sum_{i=0}^{8} a_ix_i + d = y + d\).
\(y' - y = d > 0\)

We can conclude that increasing \(x_6\) will decrease y.

\section*{Question 3:}
In linear regression models, we consider two data points \((\bm{x}_1, y_1)\) and \((\bm{x}_2, y_2)\) with \(\bm{x}_1, \bm{x}_2 \in \R^2\)
and \(y_1, y_2 \in \R\). For simplicity, we set the bias term \(b = 0\). Let \(\bm{X} \in \R^{2 \times 2}\) have rows \(\bm{x}_1^T\) and \(\bm{x}_2^T\),
and let \(\bm{y} = \begin{bmatrix}
    y_1 \\
    y_2 \\
\end{bmatrix} \in \R^2\). Assume the columns of \(\bm{X}\), denoted by \(\bm{x}^{(1)}\) and \(\bm{x}^{(2)}\), are linearly dependent such that
\(\bm{x}^{(1)} = 2\bm{x}^{(2)}\).

(a) Consider the least squares estimation:
\begin{align}
    \text{min}_{\beta \in \R^2}\|\bm{X}\beta - \bm{y}\|_2^2 
\end{align}
What problem does the linear dependency among the columns of \(\bm{X}\) cause when estimating \(\bm{\beta}\) using least squares?

(b) Now consider the ridge regression, which incorporates a regularization term:

\begin{align}
    \text{min}_{\beta \in \R^2}\|\bm{X}\bm{\beta} - \bm{y}\|_2^2 + \lambda\|\bm{\beta}\|_2^2,
\end{align}

where \(\lambda > 0\) is a regularization parameter. Derive the solution \(\hat{\bm{\beta}}\) of (2). What is the ratio between \(\hat{\beta}_1\) adn \(\hat{\beta}_2\)?

(c) Discuss how varying the value of \(\lambda\) affects the solution and its ability to mitigate issues arising from linear dependency of columns of \(\bm{X}\).

\subsection*{Answer}
todo

\section*{Question 4:}
Let \(\left\{(\bm{x}_i, y_i)\right\}_{i=1}^N\) be given with \(\bm{x}_i \in \R^n\) and \(y_i \in \R\). Consider the soft-SVM:
\begin{align*}
    \min_{a\in\R^n, b\in \R} \sum_{i=1}^N h(y_i(\langle \bm{a}, \bm{x}_i\rangle) + b) - 1 + \lambda\|\bm{a}\|_2^2,
\end{align*}
where \(\lambda \in \R\) is a regularization parameter and \(h(t) = \max\{0, -t\}\) is the hinge loss function. Prove that solving the above soft-SVM is equivalent to solving the following problem:
\begin{align*}
    &\min_{\bm{a} \in R^n, b \in \R, \bm{\xi} \in \R^N} \sum_{i = 1}^N \xi + \lambda\|\bm{a}\|_2^2, \\
    &\text{s.t. } y_i(\langle \bm{a}, \bm{x}_i\rangle + b) \geq 1 - \xi_i \text{ and } \xi_i \geq 0, i = 1,2,\cdots,N
\end{align*}

\subsection*{Answer}

\section*{Question 5:}
Let \(V\) be a Hilbert space. Let \(S_1\) and \(S_2\) be two hyperplanes in \(V\) defined by
\begin{align*}
    S_1 = \{x \in V | \langle \bm{a}_1, \bm{x}\rangle = b_1\}, S_2 = {\bm{x} \in V | \langle \bm{a}_2, \bm{x} \rangle = b_2}.
\end{align*}

Assume \(S_1 \cap S_2\) is non-empty. Let \(\bm{y} \in V\) be given. We consider the projection of \(\bm{y}\) onto \(S_1 \cap S_2\), i.e.,
the solution of 
\begin{align}
    \min_{\bm{x} \in S_1 \cap S_2}\|\bm{x} - \bm{y}\|.
\end{align}
(a) Prove that \(S_1 \cap S_2\) is a plane, i.e., if \(\bm{x}, \bm{z} \in S_1 \cap S_2\), then \((1 + t)\bm{z} - t\bm{x} \in S_1 \cap S_2\) for any \(t \in \R\). 

(b) Prove that \(\bm{z}\) is a solution of (3) if and only if \(\bm{z} \in S_1 \cap S_2\) and 
\begin{align}
    \langle \bm{z} - \bm{y}, \bm{z} - \bm{x}\rangle = 0, \forall \bm{x} \in S_1 \cap S_2
\end{align}

(c) Find and explicit solution of (3). \\

(d) Prove the solution found in part (c) is unique. \\


\end{document}