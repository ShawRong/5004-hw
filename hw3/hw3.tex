\documentclass[a4paper,12pt]{article}
\usepackage{amsmath}
\usepackage{amsfonts}
\usepackage{amssymb}
\usepackage{geometry}
\usepackage{bm}

\newcommand{\R}{\mathbb{R}}
\newcommand{\N}{\mathbb{N}}

% 设置页边距
\geometry{
    left=2cm,
    right=2cm,
    top=2cm,
    bottom=2cm,
}

\begin{document}

\section*{Question 1:}
Determine whether each of the following scalar-valued functions of \text{n-vectors} is linear. If it is a linear function, give its inner product representation, ie.,
an n-vector \(\bm{a}\) for which \(f(\bm{x}) = \bm{a}^T\bm{x}\) for all \(\bm{x}\). If it is not linear, give specific \(\bm{x}, \bm{y}\), \(\alpha \text{ and } \beta\) such that
\begin{align*}
    f(\alpha \bm{x} + \beta \bm{y}) \neq \alpha f(\bm{x}) + \beta f(\bm{y}).
\end{align*}

(a) The spread of values of the vector, defined as \(f(\bm{x}) = max_kx_k - min_kx_k \). \\
(b) The difference of the last element and the first, \(f(\bm{x}) = x_n - x_1\). \\

\section*{Answer :}
(a)
Take \(\bm{x} = (1, 2, 3)\) and \(\alpha = 1, \beta = 1\) for example:
\begin{align*}
    f(\bm{x}) &= 3 - 1 = 2 \\
    f(\bm{-x}) &= -1 + 3 = 2 \\
    f(\bm{0}) &= 0 - 0 = 0 \\
    f(\bm{x} + (-\bm{x})) &= f(\bm{0}) = 0 \\
    f(\bm{x}) + f(-\bm{x}) &= 2 + 2 = 4 \\
    f(\bm{x} + (-\bm{x})) &\neq f(\bm{x}) + f(-\bm{x}) \\
\end{align*}


\end{document}