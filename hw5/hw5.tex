
\documentclass[a4paper,12pt]{article}
\usepackage{amsmath}
\usepackage{amsfonts}
\usepackage{amssymb}
\usepackage{geometry}
\usepackage{bm}

\newcommand{\R}{\mathbb{R}}
\newcommand{\N}{\mathbb{N}}

% 设置页边距
\geometry{
    left=2cm,
    right=2cm,
    top=2cm,
    bottom=2cm,
}

\begin{document}

\title{5004 Homework 2}
\author{RONG Shuo}
\date{\today}
\maketitle


\section*{Question 1:}
1. Let \(T: \R^3 \to \R^2\) be a linear transformation. If \(T\) satisfies 
\begin{align*}
    T\begin{bmatrix}
        1 \\
        0 \\
        -1 
    \end{bmatrix} = \begin{bmatrix}
        2 \\
        3 
    \end{bmatrix}
    \text{ , and }  
    T\begin{bmatrix}
        2 \\
        1 \\
        3
    \end{bmatrix} = \begin{bmatrix}
        -1 \\
        0
    \end{bmatrix},
\end{align*} then find 
\begin{align*}
    T\begin{bmatrix}
        8 \\
        3 \\
        7
    \end{bmatrix}
\end{align*}.

\subsection*{Answer}
We know that
\begin{align*}
    2\begin{bmatrix}
        1 \\
        0 \\
        -1 \\
    \end{bmatrix}  + 3\begin{bmatrix}
        2 \\
        1 \\
        3\\
    \end{bmatrix} = \begin{bmatrix}
        2 \\ 
        0 \\
        -2\\
    \end{bmatrix} + \begin{bmatrix}
        6 \\
        3 \\
        9 \\
    \end{bmatrix} = \begin{bmatrix}
        8 \\
        1 \\
        7 \\
    \end{bmatrix}
\end{align*}
So in conclusion,
\begin{align*}
    2T\begin{bmatrix}
        1 \\
        0 \\
        -1 \\
    \end{bmatrix} +
    3T\begin{bmatrix}
        2 \\
        1 \\
        3 \\
    \end{bmatrix} = 2\begin{bmatrix}
        2 \\
        3 \\
    \end{bmatrix} + 
    3\begin{bmatrix}
        -1 \\
        0 \\
    \end{bmatrix} = \begin{bmatrix}
        4 - 3    \\
        6 + 0 \\
    \end{bmatrix} =
    \begin{bmatrix}
        1    \\
        6  \\
    \end{bmatrix}
\end{align*}

\section*{Question 2:}
2. Find the Jacobian matrix of the following vector-valued multi-variable functions. \\
(a) \(f:\R^n \to \R^m \) is defined by \(f(\bm{x}) = \bm{A}\bm{x} - \bm{b}\), where \(\bm{x} \in \R^n\), \(\bm{A} \in \R^{m \times n}, \bm{b} \in \R^n\). \\
(b) \(f: R^n \to \R^n\) is defined by \(f(\bm{x}) = \bm{x}\bm{x}^T\bm{a}\), where \(\bm{x} \in \R^n\), \(\bm{a} \in \R^n\).

\subsection*{Answer}


\section*{Question 3:}
3. Let \(f : \R^2 \to \R\), \(g : \R^2 \to \R^2\), \(g(x, y) = (x^2y, x-y)\) and \(h = f \circ g = f(g(x, y))\). Find \(\frac {\partial h}{\partial x}|_{x=1, y=2}\) if
\(\frac{\partial f}{\partial x}|_{x=2, y=-1} = 3\) and \(\frac{\partial f}{\partial y}|_{x=2, y=-1} = -2\). (Hint: use the chain rule)

\subsection{Answer :}


\section*{Question 4:}
Let \(f(t) = f_1(t) * f_2(t)\) be the convolution of two functions \(f_1(t)\) and \(f_2(t)\) on \(\R\), i.e.,
\begin{align*}
    f(t) = \int_{-\infty}^{+\infty} f_1(t - s)f_2(s)ds
\end{align*}
Let \(a, a_1, a_2\) be real number.

(i) Prove the following identity:
\begin{align*}
    f_1(t-a)*f_2(t) = f_1(t) * f_2(t-a) = f(t-a).
\end{align*}
(ii) Prove the following identity:
\begin{align*}
    f_1(t-a_1) * f_2(t-a_2) = f(t-a_1-a_2).
\end{align*}

\section*{Question 5:}
5. Let \(V_1\) and \(V_2\) be two Hilbert spaces with the inner products \(\langle \cdot, \cdot \rangle_{V_1}\), and \(\langle \cdot, \cdot\rangle_{V_2}\), respectively.
Let \(T \in \mathcal{L}(V_1, V_2)\), i.e., \(T: V_1 \to V_2\) be a bounded linear operator. \\
(a) Let \(S:V_2 \to V_1\) be an operator satisfying \(\langle T\bm{x}, \bm{y}\rangle_{V_2} = \langle \bm{x}, S\bm{y}\rangle_{V_1}\) for any \(\bm{x} \in V_1\) and \(\bm{y} \in V_2\).
Prove that \(S\) is a bounded linear operator. (COnsequently, \(S\) is the adjoint of \(T\), i.e., \(S = T^*\)) \\
(b) Prove that \((T^*)^* = T\). \\
(c) Prove that \(\|T\| = \|T^*\|\).


\subsection*{Answer}

\section*{Question 6:}
Consider the vector space \(\ell_{\infty}\) equipped with the norm \(\|\cdot\|_{\infty}\). Define the operator \(T: \ell_{\infty} \to \ell_{\infty}\) by \(T(\{x_n\}_{n \in N}) = \{y_n\}_{n\in N}\) where \(y_n = x_{n+1}\). \\
(a) Prove that \(T\) is a linear operator. \\
(b) Prove that \(T\) is a bounded operator. \\
(c) Prove that \(\|T\|=1\).

\subsection*{Answer}
\end{document}