
\documentclass[a4paper,12pt]{article}
\usepackage{amsmath}
\usepackage{amsfonts}
\usepackage{amssymb}
\usepackage{geometry}
\usepackage{bm}

\newcommand{\R}{\mathbb{R}}
\newcommand{\N}{\mathbb{N}}

% 设置页边距
\geometry{
    left=2cm,
    right=2cm,
    top=2cm,
    bottom=2cm,
}

\begin{document}

\title{5004 Homework 2}
\author{RONG Shuo}
\date{\today}
\maketitle


\section*{Question 1:}
1. Let \(T: \R^3 \to \R^2\) be a linear transformation. If \(T\) satisfies 
\begin{align*}
    T\begin{bmatrix}
        1 \\
        0 \\
        -1 
    \end{bmatrix} = \begin{bmatrix}
        2 \\
        3 
    \end{bmatrix}
    \text{ , and }  
    T\begin{bmatrix}
        2 \\
        1 \\
        3
    \end{bmatrix} = \begin{bmatrix}
        -1 \\
        0
    \end{bmatrix},
\end{align*} then find 
\begin{align*}
    T\begin{bmatrix}
        8 \\
        3 \\
        7
    \end{bmatrix}
\end{align*}.

\subsection*{Answer}
We know that
\begin{align*}
    2\begin{bmatrix}
        1 \\
        0 \\
        -1 \\
    \end{bmatrix}  + 3\begin{bmatrix}
        2 \\
        1 \\
        3\\
    \end{bmatrix} = \begin{bmatrix}
        2 \\ 
        0 \\
        -2\\
    \end{bmatrix} + \begin{bmatrix}
        6 \\
        3 \\
        9 \\
    \end{bmatrix} = \begin{bmatrix}
        8 \\
        1 \\
        7 \\
    \end{bmatrix}
\end{align*}
So in conclusion,
\begin{align*}
    2T\begin{bmatrix}
        1 \\
        0 \\
        -1 \\
    \end{bmatrix} +
    3T\begin{bmatrix}
        2 \\
        1 \\
        3 \\
    \end{bmatrix} = 2\begin{bmatrix}
        2 \\
        3 \\
    \end{bmatrix} + 
    3\begin{bmatrix}
        -1 \\
        0 \\
    \end{bmatrix} = \begin{bmatrix}
        4 - 3    \\
        6 + 0 \\
    \end{bmatrix} =
    \begin{bmatrix}
        1    \\
        6  \\
    \end{bmatrix}
\end{align*}

\section*{Question 2:}
2. Find the Jacobian matrix of the following vector-valued multi-variable functions. \\
(a) \(f:\R^n \to \R^m \) is defined by \(f(\bm{x}) = \bm{A}\bm{x} - \bm{b}\), where \(\bm{x} \in \R^n\), \(\bm{A} \in \R^{m \times n}, \bm{b} \in \R^n\). \\
(b) \(f: \R^n \to \R^n\) is defined by \(f(\bm{x}) = \bm{x}\bm{x}^T\bm{a}\), where \(\bm{x} \in \R^n\), \(\bm{a} \in \R^n\).

\subsection*{Answer}
Assume that:
\begin{align*}
    \bm{A}_i &= [A_{ij}]_{j=1}^n, \bm{A}_i \in \R^n\\
    b_i &= \bm{b}_i, b_i \in \R  \\
    f_i(\bm{x}) &= \langle \bm{A}_i, \bm{x} \rangle - b_i 
\end{align*}
We know that:
\begin{align*}
    f(\bm{x}) = \bm{A}\bm{x}-\bm{b} \\
    f(\bm{x}) = \begin{bmatrix}f_1(\bm{x}) \\ \vdots \\ f_n(\bm{x})\end{bmatrix} \\
    Df(\bm{x}) = \begin{bmatrix}\nabla f_1(\bm{x})^T \\ \vdots \\ \nabla f_n(\bm{x})^T\end{bmatrix} \\
    Df(\bm{x}) = \begin{bmatrix}\nabla f_1(\bm{x})^T \\ \vdots \\ \nabla f_n(\bm{x})^T\end{bmatrix} \\
    Df(\bm{x}) = \begin{bmatrix}\bm{A}_1^T \\ \vdots \\ \bm{A}_n^T\end{bmatrix} = \bm{A} \\ 
\end{align*}
We know that Jacobian matrix is the differentiation of \(f:\R^n \to \R^m\).

\noindent
So In conclusion, the Jacobian matrix is \(\bm{A}\).

\vspace{1cm}
\noindent
(b) 
Assume that:
\begin{align*}
    \bm{X} &= \bm{x} \bm{x}^T= [x_i x_j]_{i = 1, j = 1}^{n, n}, \bm{X} \in \R^{n \times n} \\ 
    \bm{X}_i &= [x_ix_j]_{j=1}^n, \bm{X}_i \in \R^n\\
    f_i(\bm{x}) &= \langle \bm{X}_i, \bm{a} \rangle \\
\end{align*}
if \(i = k\)
\begin{align*}
   \frac{\partial f_i}{x_k} = \sum_{j=1}^n x_ja_j + x_ka_k
\end{align*}
if \(i \ne k\)
\begin{align*}
    \frac{\partial f_i}{\partial x_k}  = x_ia_k
\end{align*}
\begin{align*}
    f(\bm{x}) &= \begin{bmatrix}
        f_1(\bm{x}) \\
        \vdots \\
        f_n(\bm{x})
    \end{bmatrix} \\
    Df(\bm{x}) &= \begin{bmatrix}
        \nabla f_1(\bm{x})^T \\
        \vdots \\
        \nabla f_n(\bm{x})^T     
    \end{bmatrix} = 
    \begin{bmatrix}
        x_1a_1 & x_1a_2 & \cdots & x_1a_n \\
        \vdots & \vdots & \cdots & \vdots\\ 
        x_na_1 & x_na_2 & \cdots & x_na_n \\
    \end{bmatrix} + \begin{bmatrix}
        \sum_{j=1}^{n}x_ja_j & 0 & \cdots & 0 \\
        \vdots & \vdots & \cdots & \vdots\\ 
        0 & 0 & \cdots &\sum_{j=1}^{n}x_ja_j 
    \end{bmatrix} = \bm{x}\bm{a}^T + (\bm{x}^T\bm{a})\bm{I}
\end{align*}




\section*{Question 3:}
3. Let \(f : \R^2 \to \R\), \(g : \R^2 \to \R^2\), \(g(x, y) = (x^2y, x-y)\) and \(h = f \circ g = f(g(x, y))\). Find \(\frac {\partial h}{\partial x}|_{x=2, y=-1}\) if
\(\frac{\partial f}{\partial x}|_{x=2, y=-1} = 3\) and \(\frac{\partial f}{\partial y}|_{x=2, y=-1} = -2\). (Hint: use the chain rule)

\subsection*{Answer :}
We know that:
\begin{align*}
    g_1(x, y) &= x^2y \\
    g_2(x, y) &= x - y \\
    h(x, y) &= f(g_1(x, y), g_2(x, y)) \\
    \frac{\partial h}{\partial x} &= \frac{\partial f}{\partial x} \frac{\partial g_1}{\partial x} + \frac{\partial f}{\partial y} \frac{\partial g_2}{\partial x} \\
    \frac{\partial h}{\partial x}|_{x=1, y=2} &= 3\frac{\partial g_1}{\partial x} -2\frac{\partial g_2}{\partial x} = 3 * (2xy)|_{x=2, y=-1} - 2 * 1 = -14
\end{align*}


\section*{Question 4:}
Let \(f(t) = f_1(t) * f_2(t)\) be the convolution of two functions \(f_1(t)\) and \(f_2(t)\) on \(\R\), i.e.,
\begin{align*}
    f(t) = \int_{-\infty}^{+\infty} f_1(t - s)f_2(s)ds
\end{align*}
Let \(a, a_1, a_2\) be real number.

\vspace{1cm}
\noindent
(i) Prove the following identity:
\begin{align*}
    f_1(t-a)*f_2(t) = f_1(t) * f_2(t-a) = f(t-a).
\end{align*}

\noindent
(ii) Prove the following identity:
\begin{align*}
    f_1(t-a_1) * f_2(t-a_2) = f(t-a_1-a_2).
\end{align*}


\subsection*{Answer :}
(i) 
Prove \(f_1(t - a) * f_2(t) = f(t - a)\):
\begin{align*}
    f(t - a) &= \int_{-\infty}^{+\infty}f_1(t - s - a)f_2(s)ds \\
    f_1'(t) &= f_1(t - a) \\
    f(t - a) &= \int_{-\infty}^{+\infty}f_1'(t - s)f_2(s)ds = f_1'(t)*f_2(t) = f_1(t - a)*f_2(t) \\
\end{align*}
Prove \(f_1(t) * f_2(t - a) = f(t - a)\):
\begin{align*}
    f(t - a) &= \int_{-\infty}^{+\infty}f_1(t - s - a)f_2(s)ds \\
    f_2'(t) &= f_2(t - a) \\
    f(t - a) &= \int_{-\infty}^{+\infty}f_1(t - (s+a))f_2(s)d(s + a) = \int_{-\infty}^{+\infty}f_1(t - (s+a))f_2'(s + a)d(s + a) \\
    &= \int_{-\infty}^{+\infty}f_1(t - u)f_2'(u)du \\
    &= f_1(t)*f_2'(t)  \\
    &= f_1()*f_2(t-a) \\
\end{align*}

\vspace{1cm}
\noindent
(ii)
\begin{align*}
    f_1(t-a_1) * f_2(t-a_2) &= \int_{-\infty}^{+\infty}f_1(t-a_1 - s)f_2(s-a_2)ds\\
    &= \int_{-\infty}^{+\infty}f_1(t-a_1 - s)f_2(s-a_2)d(s - a_2) \\
    &= \int_{-\infty}^{+\infty}f_1(t-a_1 - a_2 - (s - a_2))f_2(s-a_2)d(s - a_2) \\
    &= \int_{-\infty}^{+\infty}f_1(t-(a_1 + a_2) - u)f_2(u)du \\
    &=f(t-a_1-a_2).
\end{align*}

\section*{Question 5:}
5. Let \(V_1\) and \(V_2\) be two Hilbert spaces with the inner products \(\langle \cdot, \cdot \rangle_{V_1}\), and \(\langle \cdot, \cdot\rangle_{V_2}\), respectively.
Let \(T \in \mathcal{L}(V_1, V_2)\), i.e., \(T: V_1 \to V_2\) be a bounded linear operator. \\
(a) Let \(S:V_2 \to V_1\) be an operator satisfying \(\langle T\bm{x}, \bm{y}\rangle_{V_2} = \langle \bm{x}, S\bm{y}\rangle_{V_1}\) for any \(\bm{x} \in V_1\) and \(\bm{y} \in V_2\).
Prove that \(S\) is a bounded linear operator. (Consequently, \(S\) is the adjoint of \(T\), i.e., \(S = T^*\)) \\
(b) Prove that \((T^*)^* = T\). \\
(c) Prove that \(\|T\| = \|T^*\|\).


\subsection*{Answer}
(a)We know
\begin{align}
    \|\bm{A}\bm{x}\|_{V_2} \leq \|\bm{A}\|\|\bm{x}\|_{V_1}
\end{align}
We know, by using Cauchy–Schwarz inequality and (1):
\begin{align*}
    \langle \bm{T}\bm{x}, \bm{y}\rangle_{V_2}  \leq \|\bm{T}\bm{x}\|_{V_2}\|\bm{y}\|_{V_2} \leq \|\bm{T}\|\|\bm{x}\|_{V_1}\|\bm{y}\|_{V_2} \\
\end{align*}
We choose \(\bm{x}\) as \(\bm{S}\bm{y}\)
\begin{align*}
    \langle \bm{T}\bm{S}\bm{y}, \bm{y}\rangle_{V_2} &= \langle \bm{S}\bm{y}, \bm{S}\bm{y}\rangle_{V1} = \|\bm{S}\bm{y}\|_{V_1}^2   \leq \|\bm{T}\bm{S}\bm{y}\|_{V_2}\|\bm{y}\|_{V_2} \leq \|\bm{T}\|\|\bm{S}\bm{y}\|_{V_1}\|\bm{y}\|_{V_2} \\
    \|\bm{S}\bm{y}\|_{V_1}^2 &\leq \|\bm{T}\|\|\bm{S}\bm{y}\|_{V_1}\|\bm{y}\|_{V_2}  \\
    \|\bm{S}\bm{y}\|_{V_1} &\leq \|\bm{T}\|\|\bm{y}\|_{V_2}  \leq \infty
\end{align*}
Thus, S is a bounded linear operator.

\vspace{1cm}
\noindent
(b) 
We know:
\begin{align*}
    \langle \bm{T}\bm{x}, \bm{y}\rangle_{V_2} &= \langle \bm{x}, \bm{T}^*\bm{y}\rangle_{V_1} 
\end{align*}
\begin{align*}
    \langle \bm{x}, \bm{T}^*\bm{y}\rangle_{V_1} &= \langle (\bm{T}^*)^*\bm{x}, \bm{y}\rangle_{V_2} \\
    \langle \bm{T}\bm{x}, \bm{y}\rangle_{V_2} &= \langle (\bm{T}^*)^*\bm{x}, \bm{y}\rangle_{V_2} \\
    \bm{T} &= (\bm{T}^*)^*
\end{align*}

\vspace{1cm}
\noindent
(c)
Proof of 
\begin{align*}
    \|\bm{T}\| = \sup_{\|\bm{y}\| = 1}\|\bm{T}\bm{y}\| = \sup_{\|\bm{y}\| = 1}\sup_{\|\bm{x}\| = 1}\langle \bm{x}, \bm{T}\bm{y} \rangle
\end{align*}
By using Cauchy Schwarz inequality
\begin{align*}
    \langle \bm{x}, \bm{T}\bm{y} \rangle &\leq \|\bm{x}\|\|\bm{T}\bm{y}\| \\
    \|\bm{x}\| &= 1 \\
    \langle \bm{x}, \bm{T}\bm{y} \rangle &\leq \|\bm{T}\bm{y}\| \\
    \|\bm{T}\bm{y}\| &= \sup_{\|x\| = 1} \langle \bm{x}, \bm{T}\bm{y} \rangle \\
    \|\bm{T}\| &= \sup_{\|\bm{y}\| = 1}\|\bm{T}\bm{y}\| = \sup_{\|\bm{y}\| = 1}\sup_{\|\bm{x}\| = 1}\langle \bm{x}, \bm{T}\bm{y} \rangle
\end{align*}
Then
\begin{align*}
    \|\bm{T}^*\| &= \sup_{\|\bm{y}\|=1} \|\bm{T}^* \bm{y}\| = \sup_{\|\bm{y}\|=1}  \sup_{\|\bm{x}\|=1}\langle \bm{x}, \bm{T}^*\bm{y}\rangle \\
    &= \sup_{\|\bm{y}\| = 1} \sup_{ \|\bm{x}\| = 1} \langle \bm{T}\bm{x}, \bm{y} \rangle = \sup_{\|\bm{x}\| = 1} \sup_{ \|\bm{y}\| = 1} \langle \bm{T}\bm{x}, \bm{y} \rangle = \sup_{\|\bm{x}\|=1}\|\bm{T}\bm{x}\| = \|\bm{T}\|
\end{align*}
In conclusion, \(\|\bm{T}^*\| = \|\bm{T}\|\)

\section*{Question 6:}
Consider the vector space \(\ell_{\infty}\) equipped with the norm \(\|\cdot\|_{\infty}\). Define the operator \(T: \ell_{\infty} \to \ell_{\infty}\) by \(T(\{x_n\}_{n \in N}) = \{y_n\}_{n\in N}\) where \(y_n = x_{n+1}\). \\
(a) Prove that \(T\) is a linear operator. \\
(b) Prove that \(T\) is a bounded operator. \\
(c) Prove that \(\|T\|=1\).

\subsection*{Answer}
(a) Additivity: Proof: \\
For any \(\{x_n\}_{n \in \N} , \{z_n\}_{n \in \N} \in \ell_\infty\)
\begin{align*}
    &T(\{x_n\}_{n\in\N} + \{z_n\}_{n\in\N}) = T(\{x_n + z_n\}_{n\in\N}) = \\
    &\{(x_n + z_n)_{n+1}\}_{n\in\N} = \{x_{n+1} + z_{n+1}\}_{n\in\N} = \{x_{n+1}\}_{n\in\N} + \{z_{n+1}\}_{n\in\N} = T(\{x_n\}_{n\in\N}) + T(\{z_n\}_{n\in\N})
\end{align*}

Homogeneity Proof: \\
\begin{align*}
    T(\alpha \{x_n\}_{n \in \N}) = T(\{\alpha x_n\}_{n \in \N}) = \{\alpha x_{n+1}\}_{n \in \N} = \alpha \{x_{n+1}\}_{n \in \N} = \alpha T(\{x_n\}_{n \in \N})
\end{align*}

\vspace{1cm}
\noindent
(b) 
To prove \(T\) is bounded operator, we can prove, for a constance \(C\), that :
\begin{align*}
    &\|T(\{x_n\}_{n \in \N})\|_\infty \leq C\|\{x_n\}_{n \in \N}\|_\infty , \forall \{x_n\}_{n \in \N} \in \ell_\infty\\
    &\|T(\{x_n\}_{n \in \N})\|_\infty = \sup_{n\in\N}|y_n| = \sup_{n\in\N}|x_{n+1}| \leq \sup_{n\in\N}|x_n| = \|\{x_n\}_{n\in\N}\|_{\infty} \\
    &\|T(\{x_n\})\|_\infty \leq \|\{x_n\}\|_\infty
\end{align*}
In conclusion, \(T\) is a bounded operator.


\vspace{1cm}
\noindent
(c)
From (b), we know that \(\|T(\{x_n\})\|_\infty \leq \|\{x_n\}\|_\infty\), s.t. \(\|T\| \leq 1\). \\
Consider the sequence \(\{x_n\}_{n \in \N} = \{ 1, 1, \cdots\} = \{x_{n+1}\}_{n\in\N}\)
By definition, we know:
\begin{align*}
    &\|T\| \geq \frac{\|T(\{x_n\})\|_\infty}{\|\{x_n\}\|_\infty} = 1 \\
    &1\leq\|T\| \leq 1
\end{align*}
Thus, we conclude that \(\|T\| = 1\)

\end{document}