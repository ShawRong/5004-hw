\documentclass[a4paper,12pt]{article}
\usepackage{amsmath}
\usepackage{amsfonts}
\usepackage{amssymb}
\usepackage{geometry}
\usepackage{bm}

\newcommand{\R}{\mathbb{R}}
\newcommand{\N}{\mathbb{N}}

% 设置页边距
\geometry{
    left=2cm,
    right=2cm,
    top=2cm,
    bottom=2cm,
}

\begin{document}

\title{5004 Homework 2}
\author{RONG Shuo}
\date{\today}
\maketitle

\section*{Question: 1}
1. Let \((V, \|\cdot\|)\) be a normed vector space.

(a) Prove that, for all \(\bm{x}, \bm{y} \in V\),
\[
    |\|\bm{x}\| - \|\bm{y}\|| \leq \|\bm{x} - \bm{y}\|.
\]

(b) Let \(\{\bm{x}_k\}_{k\in \N}\) be a convergent sequence in \(V\) with limit \(\bm{x} \in V\). Prove that
\[
    \lim_{k\to\infty}\|\bm{x}_k\| = \|\bm{x}\|.
\]

\textit{(Hint: Use part (a).)}

\subsection*{Answer (a):}
We begin by recalling two fundamental properties of norms:
\begin{align}
    \|\bm{x} + \bm{y}\| \leq \|\bm{x}\| + \|\bm{y}\| \\
    \|-\bm{x}\| = \|\bm{x}\| 
\end{align}
For two vectors, \(\bm{x - y}, \bm{y}\), according to (1), we get:
\begin{align*}
    \|\bm{x} - \bm{y} + \bm{y}\| \leq \|\bm{x} - \bm{y}\| + \|\bm{y}\|  \\
    \|\bm{x}\| \leq \|\bm{x} - \bm{y}\| + \|\bm{y}\|  \\
    \|\bm{x}\| - \|\bm{y}\| \leq \|\bm{x} - \bm{y}\| 
\end{align*}
Similarly, for two vectors, \(\bm{y -x}, \bm{x}\), we have:
\begin{align*}
    \|\bm{y} - \bm{x} + \bm{x}\| \leq \|\bm{y} - \bm{x}\| + \|\bm{x}\|  \\
    \|\bm{y}\| \leq \|\bm{y} - \bm{x}\| + \|\bm{x}\|  \\
    \|\bm{y}\| - \|\bm{x}\| \leq \|\bm{y} - \bm{x}\| 
\end{align*}
We know (2), s.t.:
\begin{align*}
    \|\bm{y}\| - \|\bm{x}\| \leq \|\bm{x} - \bm{y}\| 
\end{align*}
Because of:
\begin{align*}
    \|\bm{x}\| - \|\bm{y}\| \leq \|\bm{x} - \bm{y}\|  \\
    \|\bm{y}\| - \|\bm{x}\| \leq \|\bm{x} - \bm{y}\|  \\
\end{align*}
We can conclude:
For all \(\bm{x}, \bm{y} \in V\),
\begin{align}
    |\|\bm{x}\| - \|\bm{y}\|| \leq \|\bm{x} - \bm{y}\|
\end{align}

\subsection*{Answer (b):}
We know (3), and \(0 \leq |x|, \forall x \in \R\), s.t.
\begin{align}
    0 \leq |\|\bm{x}\| - \|\bm{y}\|| \leq \|\bm{x} - \bm{y}\| 
\end{align}
Since we know \(\bm{x}_k \to \bm{x}\), we have:
\begin{align}
    \lim_{k \to \infty}\|\bm{x}_k - \bm{x}\| = 0
\end{align}
According to (4), (5), we have:
\begin{align*}
    0 \leq |\|\bm{x}_k\| - \|\bm{x}\|| \leq \|\bm{x}_k - \bm{x}\| \\
    0 \leq \lim_{k \to \infty}|\|\bm{x}_k\| - \|\bm{x}\|| \leq \lim_{k \to \infty}\|\bm{x}_k - \bm{x}\| \\
    0 \leq \lim_{k \to \infty}|\|\bm{x}_k\| - \|\bm{x}\|| \leq 0\\
    \lim_{k \to \infty}|\|\bm{x}_k\| - \|\bm{x}\|| = 0  \\
    \lim_{k \to \infty}\|\bm{x}_k\| = \|\bm{x}\|
\end{align*}
\section*{Question 2:}

2. Let \(V\) be a vector space and \(\{\bm{a}_1,\bm{a}_2,\cdots,\bm{a}_n\}\) be a basis of \(V\). If \(
\bm{u} = u_1\bm{a}_1 + \cdots + u_n\bm{a}_n\) and \(\bm{v} = v_1\bm{a}_1 + \cdots + v_n\bm{a}_n\) are two vectors in \(V\), define

\[
    \langle \bm{u}, \bm{v}\rangle = u_1v_1 + \cdots + u_nv_n.
\]

Show that this is an inner product on V.

\subsection*{Answer:}
\subsubsection*{Positive Definite Property:}
For any \(\bm{u} \in V\), we know:
\begin{align}
    u_k^2 \geq 0, \forall u_k \in \R
\end{align}
s.t.
\begin{align*}
    \langle\bm{u}, \bm{u} \rangle  = u_1^2 + \cdots + u_n^2 \geq 0 
\end{align*}
For any \(\bm{u} = \bm{0}\) , we have:
\begin{align*}
    \langle \bm{u},\bm{u} \rangle = 0 + \cdots + 0 = 0
\end{align*}
For any \(\langle\bm{u}, \bm{u}\rangle = 0\) and (6), we have:
\begin{align*}
    \langle\bm{u}, \bm{u}\rangle = 0 \\
    u_1^2 + \cdots + u_n^2 = 0
\end{align*}

Assume for the sake of contradiction that there exists at least one \(u_j \neq 0\) for some \(j \in \{1, 2, \cdots, n\}\).

Since \(u_j^2 > 0\), we can express it as:
\begin{align*}
    u_j^2 &= c &\text{ where } c > 0.  \\
    u_1^2 + u_2^2 + \cdots + u_n^2 &= u_1^2 + u_2^2 + \cdots + u_{j - 1}^2 + c + u_{j + 1}^2 +\cdots +  u_n^2. \\
    u_1^2 + u_2^2 + \cdots + u_{j-1}^2 + c + u_{j+1}^2 + \cdots + u_n^2 &\geq c, &\text{ Since } u_k^2 \geq 0, \forall k \\
\end{align*}
This contradicts the initial condition. \(u_1^2 + u_2^2 + \cdots + u_n^2 = 0\).

In conclusion:
\begin{align*}
    \langle\bm{u}, \bm{u} \rangle \geq 0, \forall \bm{u} \in V. \\
    \langle\bm{u}, \bm{u}\rangle = 0 \iff \bm{u} = 0.
\end{align*}

\subsubsection*{Symmetric:}
\begin{align*}
    \langle\bm{u}, \bm{v} \rangle &= u_1v_1 + \cdots + u_nv_n \\
    u_1v_1 + \cdots + u_nv_n &= v_1u_1 + \cdots + v_nu_n \\
    \langle\bm{v}, \bm{u} \rangle &= v_1u_1 + \cdots + v_nu_n \\
    \langle\bm{u}, \bm{v} \rangle &= \langle\bm{v}, \bm{u} \rangle \\
\end{align*}
In conclusion:
\[
    \langle\bm{u}, \bm{v} \rangle = \langle\bm{v}, \bm{u} \rangle \\
\]

\subsubsection*{Linearity:}
\begin{align*}
    \langle \alpha \bm{u} + \beta \bm{v}, \bm{w}\rangle &= \sum_{i=1}^{n} (\alpha u_i + \beta v_i) w_i \\
    \sum_{i=1}^{n} (\alpha u_i + \beta v_i) w_i  &= \sum_{i=1}^{n} (\alpha u_i w_i) + (\beta v_i w_i) \\
    \sum_{i=1}^{n} (\alpha u_i w_i) + (\beta v_i w_i) &= \alpha \sum_{i=1}^{n}u_iw_i + \beta \sum_{i=1}^{n}v_iw_i\\
    \alpha \sum_{i=1}^{n}u_iw_i + \beta \sum_{i=1}^{n}v_iw_i &=  \alpha \langle\bm{u}, \bm{w}\rangle + \beta \langle\bm{v}, \bm{w}\rangle \\
\end{align*}
In conclusion:
\[
    \langle \alpha \bm{u} + \beta \bm{v}, \bm{w} \rangle =  \alpha \langle\bm{u}, \bm{w}\rangle + \beta \langle\bm{v}, \bm{w}\rangle \\
\]

\section*{Question 3:}
3. Let \(V\) be a vector space with a norm \(\|\cdot\|\) that satisfies the parallelogram identity
\[
    \|\bm{x} + \bm{y}\|^2 + \|\bm{x} - \bm{y}\|^2 = 2\|\bm{x}\|^2 + 2\|\bm{y}\|^2, \forall \bm{x}, \bm{y} \in \bm{V}.
\]
Note that we don;t have an inner product on \(V\) so far. For any \(\bm{x},\bm{y} \in V\), define
\[
    f(\bm{x}, \bm{y}) := \frac 1 2 (\|\bm{x} + \bm{y}\|^2 - \|\bm{x}\|^2 - \|\bm{y}\|^2)
\]
(a) Prove \(f(\bm{x}, \bm{x}) \geq 0\) for any \(\bm{x} \in V\), and \(f(\bm{x}, \bm{x}) = 0\) if and only if \(\bm{x} = 0\).

(b) Prove \(f(\bm{x}, \bm{y}) = f(\bm{y}, \bm{x})\) for all \(\bm{x}, \bm{y} \in V\)

(c) Prove \(f(\bm{x}+\bm{y}, \bm{z}) = f(\bm{x}, \bm{z}) + f(\bm{y}, \bm{z})\) for all \( \bm{x}, \bm{y}, \bm{z}\in V\)

(d) Prove \(f(-\bm{x}, \bm{y}) = -f(\bm{x}, \bm{y})\) for \(\bm{x}, \bm{y}\in V\)

(e) Prove \((f(\bm{x}, \bm{y}))^2 \leq f(\bm{x}, \bm{x})f(\bm{y}, \bm{y})\) for all \(\bm{x}, \bm{y} \in V\)

(c)(d)(e) together with some other technique can show that \(f(\alpha \bm{x} + \beta \bm{y}, \bm{z})  = \alpha f(\bm{x}, \bm{z}) + \beta f(\bm{y}, \bm{z})\).
Therefore, we can finally prove \(f\) defines an inner product. This question showed that the parallelogram
identity is also a sufficient condition for a norm to be induced by an inner product. Combined with
the parallelogram law on inner product spaces, we see that the parallelogram identity is a necessary
and sufficient condition for a norm to be an induced by an inner product.

\subsection*{Answer}
(a) 
\begin{align*}
    f(\bm{x}, \bm{x}) &= \frac 1 2 (\|\bm{x}+\bm{x}\|^2 - \|\bm{x}\|^2 - \|\bm{x}\|^2 ) \\
    f(\bm{x}, \bm{x}) &= \frac 1 2 (4\|\bm{x}\|^2 - \|\bm{x}\|^2 - \|\bm{x}\|^2 ) \\
    f(\bm{x}, \bm{x}) &= \frac 1 2 (2\|\bm{x}\|^2) \\
    f(\bm{x}, \bm{x}) &= \|\bm{x}\|^2 \\
    \|\bm{x}\|^2 &\geq 0 
\end{align*}
We know that:
\begin{align}
    \|\bm{x}\| = 0 \iff \bm{x} = \bm{0}
\end{align}
s.t.
\begin{align*}
    \|\bm{x}\|^2 = 0 \iff \bm{x} = \bm{0}
\end{align*}
In conclusion, \(f(\bm{x}, \bm{x}) \geq 0\) for any \(\bm{x} \in V\), and \(f(\bm{x}, \bm{x}) = 0\) if and only if \(\bm{x} = 0\).

(b) 
We know that:
\begin{align}
    \|\bm{x} + \bm{y}\| = \|\bm{y} + \bm{x}\|
\end{align}
\begin{align*}
    f(\bm{y},\bm{x}) &= \frac 1 2 (\|\bm{y} + \bm{x}\|^2 - \|\bm{y}\|^2 - \|\bm{x}\|^2)  \\
    \frac 1 2 (\|\bm{y} + \bm{x}\|^2 - \|\bm{y}\|^2 - \|\bm{x}\|^2) &= \frac 1 2(\|\bm{x} + \bm{y}\|^2 - \|\bm{x}\|^2 - \|\bm{y}\|^2) \\
    f(\bm{x}, \bm{y}) &= \frac 1 2 (\|\bm{x} + \bm{y}\|^2 - \|\bm{x}\|^2 - \|\bm{y}\|^2) \\
\end{align*}
\begin{align}
    f(\bm{x}, \bm{y}) &= f(\bm{y}, \bm{x})
\end{align}

(c)
\begin{align*}
    \|\bm{x} + \bm{y} + \bm{z}\|^2 + \|\bm{x} + \bm{y} - \bm{z}\|^2 &= 2\|\bm{x} + \bm{y}\|^2 + 2\|\bm{z}\|^2 \\
    f(\bm{x} + \bm{y}, \bm{z}) &= \frac 1 2 (\|\bm{x} + \bm{y} + \bm{z}\|^2 - \|\bm{x} + \bm{y}\|^2 - \|\bm{z}\|^2) \\
    f(\bm{x} + \bm{y}, \bm{z}) &= \frac 1 4 (\|\bm{x} + \bm{y} + \bm{z}\|^2 - \|\bm{x} + \bm{y} - \bm{z}\|^2) \tag{i}\\
\end{align*}
\begin{align*}
    & f(\bm{x}, \bm{z}) + f(\bm{y}, \bm{z}) \\
    &= \frac 1 2 (\|\bm{x} + \bm{z}\|^2 - \|\bm{x}\|^2 - \|\bm{z}\|^2) + \frac 1 2 (\|\bm{y} + \bm{z}\|^2 - \|\bm{y}\|^2 - \|\bm{z}\|^2) \\
    &= \frac 1 4 (\|\bm{ x } + \bm{z}\|^2 - \|\bm{x} - \bm{z}\|^2) + \frac 1 4 (\|\bm{y} + \bm{z}\|^2 - \|\bm{y} - \bm{z}\|^2) \tag{ ii } \\
\end{align*}
\begin{align*}
    &\|\bm{x}+\bm{y}+\bm{z}\|^2 = 2\|\bm{x} + \bm{z}\|^2 + 2\|\bm{y}\|^2 - \|\bm{x} - \bm{y} + \bm{z}\|^2 \\
    &\|\bm{x}+\bm{y}+\bm{z}\|^2 = 2\|\bm{y} + \bm{z}\|^2 + 2\|\bm{x}\|^2 - \|-\bm{x} + \bm{y} + \bm{z}\|^2 \\
    &\|\bm{x}+\bm{y}+\bm{z}\|^2 = \|\bm{x} + \bm{z}\|^2 + \|\bm{y} + \bm{z}\|^2 + \|\bm{x}\|^2 + \|\bm{y}\|^2 - \frac 1 2 (\|\bm{x} - \bm{y} + \bm{z}\|^2 + \|-\bm{x} + \bm{y} + \bm{z}\|^2)
\end{align*} 
\begin{align*} 
    &\|\bm{x}+\bm{y}-\bm{z}\|^2 = 2\|\bm{x} - \bm{z}\|^2 + 2\|\bm{y}\|^2 - \|\bm{x} - \bm{y} - \bm{z}\|^2 \\
    &\|\bm{x}+\bm{y}-\bm{z}\|^2 = 2\|\bm{y} - \bm{z}\|^2 + 2\|\bm{x}\|^2 - \|-\bm{x} + \bm{y} - \bm{z}\|^2 \\
    &\|\bm{x}+\bm{y}-\bm{z}\|^2 = \|\bm{x} - \bm{z}\|^2 + \|\bm{y} - \bm{z}\|^2 + \|\bm{x}\|^2 + \|\bm{y}\|^2 - \frac 1 2 (\|\bm{x} - \bm{y} - \bm{z}\|^2 + \|-x + \bm{y} - \bm{z}\|^2) 
\end{align*}
\begin{align*}
    &\|\bm{x} + \bm{y} + \bm{z}\|^2 - \|\bm{x} + \bm{y} -\bm{z}\|^2 = \|\bm{x} + \bm{z}\|^2 + \|\bm{y} + \bm{z}\|^2 - (\|\bm{x} - \bm{z}\|^2 + \|\bm{y} - \bm{z}\|^2 ) \\
    &= (\|\bm{x} + \bm{z}\|^2 - \|\bm{x} - \bm{z}\|^2) + (\|\bm{y} + \bm{z}\|^2 -\|\bm{y} - \bm{z}\|^2 ) \tag{iii}
\end{align*}
We know (i),(ii),(iii) s.t.
\begin{align*}
    f(\bm{x} + \bm{y}, \bm{z}) =  f(\bm{x}, \bm{z}) + f(\bm{y}, \bm{z}) \\
\end{align*}
In conclusion, \(f(\bm{x}+\bm{y}, \bm{z}) = f(\bm{x}, \bm{z}) + f(\bm{y}, \bm{z})\) for all \( \bm{x}, \bm{y}, \bm{z}\in V\)

(d)
We know(2),
\begin{align*}
    \|\bm{x} + \bm{y}\|^2 = - \|\bm{x} - \bm{y}\|^2 + 2\|\bm{x}\|^2 + 2\|\bm{y}\|^2, \forall \bm{x}, \bm{y} \in \bm{V}.
\end{align*}
s.t.
\begin{align*}
    f(\bm{x}, \bm{y}) &= \frac 1 2 (\|\bm{x} + \bm{y}\|^2 - \|\bm{x}\|^2 - \|\bm{y}\|^2) \\
    &= \frac 1 2 (- \|\bm{x} - \bm{y}\|^2 + 2\|\bm{x}\|^2 + 2\|\bm{y}\|^2 - \|\bm{x}\|^2 - \|\bm{y}\|^2 )  \\
    &= \frac 1 2 (- \|\bm{x} - \bm{y}\|^2 + \|\bm{x}\|^2 + \|\bm{y}\|^2)  \\
    &= -\frac 1 2 (\|\bm{x} - \bm{y}\|^2 - \|\bm{x}\|^2 - \|\bm{y}\|^2)  \\
    &= -\frac 1 2 (\|\bm{x} - \bm{y}\|^2 - \|\bm{x}\|^2 - \|-\bm{y}\|^2)  \\
    &= -f(\bm{x}, -\bm{y}) \\
    f(\bm{x}, \bm{y}) = -f(\bm{x}, -\bm{y})
\end{align*}
we know (9), s.t.
\begin{align*}
    f(\bm{x}, \bm{y}) = -f(\bm{x}, -\bm{y}) \equiv f(\bm{y}, \bm{x}) = -f(-\bm{y}, \bm{x})\\
    f(-\bm{y}, \bm{x}) = -f(\bm{y}, \bm{x}) \equiv f(-\bm{x}, \bm{y}) = -f(\bm{x}, \bm{y}) \\
\end{align*}
In conclusion \(f(-\bm{x}, \bm{y}) = -f(\bm{x}, \bm{y})\)

(e)
We know:
\begin{align*}
    \|\bm{x} + \bm{y}\| \leq \|\bm{x}\| + \|\bm{y}\|
\end{align*}
s.t.
\begin{align*}
    \|\bm{x} + \bm{y}\|^2 \leq \|\bm{x}\|^2 + \|\bm{y}\|^2 + 2\|\bm{x}\|\|\bm{y}\| \\
    \|\bm{x} + \bm{y}\|^2 - \|\bm{x}\|^2 - \|\bm{y}\|^2 \leq 2\|\bm{x}\|\|\bm{y}\| \\
    f(\bm{x}, \bm{y}) \leq \|\bm{x}\|\|\bm{y}\| \\
    f(\bm{x}, \bm{y})^2 \leq \|\bm{x}\|^2\|\bm{y}\|^2 \\
    f(\bm{x}, \bm{y})^2 \leq f(\bm{x}, \bm{x})f(\bm{y}, \bm{y}) \\
\end{align*}
In conclusion, \((f(\bm{x}, \bm{y}))^2 \leq f(\bm{x}, \bm{x})f(\bm{y}, \bm{y})\) for all \(\bm{x}, \bm{y} \in V\)

\section*{Question 4:}
Consider the kernel \(K(\bm{x}, \bm{y}) = e^{\bm{x}^T\bm{y}} \text{ for } \bm{x}, \bm{y} \in \R^2\). 
Find an explicit feature space \(H\) (a Hilbert space)  
and the feature map \(\phi : \R^2 \to H\)  
satisfying \(\left< \phi(\bm{x}), \phi(\bm{y})\right> = K(\bm{x}, \bm{y}) \) \\ 
\text{What is the inner product and the induced norm on \(H\)?} \\
\(H\) \text{might be infinite dimensional, and consider the Taylor's expansion} 
\(e^t = 1 + \frac{t}{1!} + \frac{t^2}{2!} + \frac{t^3}{3!} + \cdots.\)

\subsection*{Answer :}
\begin{align*}
    e^{\bm{x}^T\bm{y}} &= \sum_{n=0}^{\infty} \frac {(\bm{x}^T\bm{y})^n} {n!}
\end{align*}
And we know, the feature map of kernel function \(K(\bm{x}, \bm{y}) = \bm{x}^T\bm{y}\) is \(\phi(\bm{x}) = \bm{x}\) \\
For \(K(\bm{x}, \bm{y}) = K_1(\bm{x}, \bm{y}) \cdot K_2(\bm{x}, \bm{y})\), the feature map is \(\phi(\bm{x}) = \phi_1(\bm{x}) \otimes \phi_2(\bm{x})\) \\
As a result, we can define the feature map as
\begin{align*}
    \phi(\bm{x}) &= \left( 1, \bm{x}, \frac {1} {\sqrt{2!}} \bm{x} \otimes \bm{x}, \frac {1}{\sqrt 3!}\bm{x} \otimes \bm{x} \otimes \bm{x}, \cdots\right)   \\
    &= (1, x_1, x_2, \frac {1}{\sqrt{2!}} x_1x_1, \cdots) \\
    \text{where } \bm{x} &= (x_1, x_2), \bm{x} \otimes \bm{x} = (x_1x_1, x_1x_2, x_2x_1, x_2x_2) \cdots
\end{align*}
It is obvious that \(H\) with a standard inner product\(\left<\cdot\right>\):
\begin{align*}
    \phi(\bm{x})^T\phi(\bm{y}) = \left<\phi(\bm{x}), \phi(\bm{y})\right> = e^{\bm{x}^T\bm{y}} = K(\bm{x}, \bm{y})
\end{align*}
We know the induced norm:
\begin{align*}
    \|\phi(\bm{x})\|^2 = \left<\phi(\bm{x}), \phi(\bm{x})\right> = e^{\bm{x}^T\bm{x}}   \\
    \|\phi(\bm{x})\| = e^{\frac {\bm{x}^T\bm{x}} 2}
\end{align*}

\section*{Question 5:}
Let \(X \in \R^2\) be a two-dimensional input space, and consider the feature map: \(\phi: X \to \R^3\) defined by
\begin{align*}
    \phi(\bm{x}) = (x_1^2, x_2^2, \sqrt 2 x_1x_2),
\end{align*}
where \(\bm{x} = (x_1, x_2) \in \R^2\). We are given the function \(K\): \(\R^2 \times \R^2 \to \R\) defined by
\begin{align*}
    K(\bm{x}, \bm{y}) = \left< \phi(\bm{x}), \phi(\bm{y})\right>,
\end{align*}
where \(\left<\cdot , \cdot\right>\) denotes the standard inner product in \(\R^3\). Prove that \(K\) is a kernel function.

\subsection*{Answer:}
\begin{align*}
    K(\bm{x}, \bm{y}) &= \left< \phi(\bm{x}),\phi(\bm{y})\right> = \phi(\bm{x})^T\phi(\bm{y}) \\
    K(\bm{x}, \bm{y}) &= x_1^2y_1^2 + x_2^2y_2^2 + 2x_1x_2y_1y_2 \\
    &= (x_1y_1 + x_2y_2)^2 \\
    &= (\bm{x}^T\bm{y})^2 \\
\end{align*}
We know polynomial kernel:
\begin{align*}
    K(\bm{x}, \bm{y}) = (\bm{x}^T\bm{y}+c)^d
\end{align*}
with \(c=0\) and \(d=2\).

Polynomial kernels are known to be valid kernel functions. They satisfy the necessary properties:

1. Symmetry: \(K(\bm{x}, \bm{y}) = K(\bm{y}, \bm{x})\)
2. Positive semi-definiteness: For any finite set of points \(\{\bm{x}_1, ..., \bm{x}_n\}\), the Gram matrix \(K_{ij} = K(\bm{x}_i, \bm{x}_j)\) is positive semi-definite.

Therefore, \(K(\bm{x}, \bm{y}) = (\bm{x}^T\bm{y})^2\) is indeed a valid kernel function.

\end{document}